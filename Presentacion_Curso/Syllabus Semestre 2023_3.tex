\input{../Preambulos/preambulo_materiales}

\author{M. en C. Gustavo Contreras Mayén. \texttt{curso.fisica.comp@gmail.com}\\
M. en C. Abraham Lima Buendía. \texttt{abraham3081@ciencias.unam.mx}}
\title{Syllabus de Curso de Física Estadística \\ {\large Semestre 2023-2 Grupo 8405}}
\date{ }
% \makeatletter
% \renewcommand{\@biblabel}[1]{}
% \renewenvironment{thebibliography}[1]
%      {\section*{\refname}%
%       \@mkboth{\MakeUppercase\refname}{\MakeUppercase\refname}%
%       \list{}%
%            {\labelwidth=0pt
%             \labelsep=0pt
%             \leftmargin1.5em
%             \itemindent=-1.5em
%             \advance\leftmargin\labelsep
%             \@openbib@code
%             }%
%       \sloppy
%       \clubpenalty4000
%       \@clubpenalty \clubpenalty
%       \widowpenalty4000%
%       \sfcode`\.\@m}
% \makeatother

% \usepackage[backend=biber, style=ieee, sorting=ynt]{biblatex}
% \addbibresource{LibrosMV.bib}

\begin{document}

\renewcommand\labelenumii{\theenumi.{\arabic{enumii}}}
\maketitle
\fontsize{12}{12}\selectfont

\textbf{Lugar: } 201. Edificio Yellizcalli.
\par
\textbf{Horario: } Lunes, Miércoles y Viernes de 18 a 20 horas.
\par
\par
\section{Objetivos:}

En conformidad con los objetivos que se indican en el plan curricular de la carrera de Física:
\begin{enumerate}
\item Esta es una alternativa al punto de vista de la termodinámica, en que se presentan modelos microscópicos de sistemas de muchas partículas.
\item A partir de los postulados y las técnicas estadísticas se generan tanto la conexión conceptual con la termodinámica como las propiedades de los sistemas físicos.
\end{enumerate}

\section{Metodología de Enseñanza.}

\noindent
\textbf{Antes de la clase.}
\\
Para facilitar la discusión en el aula, el alumno revisará el material de trabajo que se le proporcionará oportunamente, así como la solución de algunos ejercicios, de tal manera que llegará a la clase conociendo el tema a desarrollar. Daremos por entendido de que el alumno realizará la lectura y actividades establecidas.
\\
\textbf{Durante la clase.}
\\
En un primer momento se tendrá la exposición con diálogo y discusión del material de trabajo con los temas a cubrir durante el semestre, posteriormente se dedicará parte de la sesión para presentar ejercicios y revisar el procedimiento de solución.
\\
\textbf{Después de la clase.}
\\
Al concluir la clase, se tendrán ejercicios a resolver, para que pueda repasar el tema visto en clase. En caso de que algún ejercicio haya quedado incompleto, deberá de resolver y entregarlos en la plataforma Moodle.

\subsection{Plataforma Moodle.}

En este semestre el curso se impartirá en modalidad presencial, se mantendrá el uso de la plataforma Moodle para el mismo, en donde se incluirán materiales de consulta: ejercicios adicionales, lecturas, artículos, enlaces a videos, a archivos para utilizar, por lo que se tendrá un apoyo adicional, de manera que contarán con herramientas adicionales.
\par
Los ejercicios que forman parte de la evaluación así como los exámenes parciales resueltos se enviarán por Moodle, se otorgorá el suficiente tiempo para la solución y entrega. No se recibirán actividades por correo electrónico.

\section{Evaluación.}

El esquema de evaluación para el curso considera los siguientes puntos:
\begin{enumerate}[label=\alph*)]
\item \textbf{Examen-Tarea $\mathbf{70\%}$} : Se tendrán tres examen-tarea durante el curso, se les proporcionará de manera adelantada y con fecha de entrega definida, no se recibirán entregas extemporáneas. 
\par
Se recomienda entregar el $100\%$ de los ejercicios. Un examen-tarea se considera acreditado cuando la calificación obtenida es mayor o igual a seis. En caso de que en alguno (o más) examen-tarea, la calificación sea menor a seis, ya se es candidato a presentar el examen final del curso.
\item \textbf{Ejercicios $\mathbf{30\%}$} : Por cada tema del curso se tendrán ejercicios para resolver, en este caso, también se recomienda entregar el mayor número de ejercicios. Además de servir para repasar el tema, les aportará puntaje para la calificación final.
\end{enumerate}

Puntaje adicional: Exposición individual de un ejemplo. Se dedicarán sesiones para la presentación en clase de un ejemplo con el formalismo correspondiente del curso. Esta evaluación aportará puntaje para la calificación final. El puntaje máximo sobre la calificación final será de 1 punto.

La calificación final del curso se obtendrá de las calificaciones de cada uno de los componentes de la evaluación: de los examen-tareas y ejercicios en clase. En el caso de obtener una calificación final mayor o igual a $6$, es la que se asentará en el acta del curso.

\section{Examen final.}

Para presentar el examen final del curso se deben de cumplir cada una de las siguientes condiciones:
\begin{enumerate}\label{ref:criterios_final}
\item Que en un examen-tarea (o más), la calificación sea menor a seis. Si los tres examen-tarea tienen calificación aprobatoria, no se permite presentar el examen final para \enquote{subir} la calificación del curso.
\item Se hayan entregado los tres examen-tarea parciales.
\end{enumerate}
En caso de que no se cumplan las condiciones anteriores, no se podrá presentar el examen final. En acuerdo con el Reglamento de Estudios Profesionales, habrá dos oportunidades para presentar el examen final, cuyas fechas se indican en el calendario del semestre 2023-3.
\par
Puntalizando sobre el examen final:
\begin{enumerate}[label=\roman*)]
\item Si en la primera ronda de examen final, la calificación obtenida es aprobatoria (mayor o igual a seis), ésta es la que se asentará en el acta del curso, ya no se promedia con los otros elementos de evaluación.
\item Si la calificación del examen final en la primera ronda es no aprobatoria, se aplicará nuevamente un examen final en la segunda ronda. La calificación obtenida en esta segunda ronda, es la que se asentará en el acta del curso.
\item Si el alumno no se presenta a la primera ronda del examen final, tendrá cinco como calificación final. Ya no podrá presentar la segunda ronda del examen final.
\end{enumerate}
\par
\textbf{Importante: } \emph{En caso de haber presentado al menos un examen-tarea y/o haber entregado al menos un ejercicio}, pero si ya no se tiene un posterior registro de entregas, se considera que abandonaron el curso, al no cumplir con los puntos de la lista del numeral \ref{ref:criterios_final}, no se podrá presentar el examen final del curso.
\par
Se asentará en el acta de calificaciones \textcolor{blue}{No Presentó (NP)}, si y solo si: el alumno no entrega ejercicio alguno y no entrega algún examen-tarea (¿?). Ocupando nuevamente el Reglamento de Estudios Profesionales, tomen en cuenta que:
\begin{itemize}
\setlength\itemsep{1pt}
\item No \enquote{se guardan calificaciones}.
\item No se renuncia a una calificación.
\end{itemize}

\section{Contenido del curso:}
Se trabajará el temario oficial de la asignatura, que está disponible en: \href{https://www.fciencias.unam.mx/sites/default/files/temario/829.pdf}{esta liga.}

Los temas del curso son:

\begin{enumerate}
\item Introducción.
\item Probabilidad en física estadística.
\item Mecánica estadística a la Gibbs.
\item Mecánica estadística cuántica.
\item La radiación de cuerpo negro.
\item Sistema de partículas interactuantes, transiciones de fase y puntos críticos.
\item Fluctuaciones.
\item Fundamentos de teoría cinética.
\item Aplicaciones modernas de la física estadística.
\end{enumerate}

\section{Fechas importantes.}

\begin{itemize}
\setlength\itemsep{1pt}
\item Lunes 30 de enero. Inicio del semestre 2023-3.
\item \textcolor{red}{Lunes 6 de febrero. Día feriado.}
\item \textcolor{red}{Lunes 20 de marzo. Día feriado.}
\item \textcolor{red}{Lunes 3 al viernes 7 de abril. Período de asueto.}
\item \textcolor{red}{Lunes 1 de mayo. Día feriado.}
\item \textcolor{red}{Miércoles 10 de mayo. Día feriado.}
\item \textcolor{red}{Lunes 15 de mayo. Día feriado.}
\item Viernes 26 de mayo. Fin de semestre 2023-3.
\item Del lunes 29 de mayo al viernes 2 de junio, primera semana de finales.
\item Del lunes 5 al viernes 9 de junio, segunda semana de finales.
\end{itemize}

\section{Bibliografía.}

Se recomienda la consulta de los siguientes textos, en cada uno de los temas se propocionará bibliografía adicional para una mejor comprensión del tema.
\nocite{*}
\printbibliography[keyword={basica}, title={Referencias Básicas}]
\printbibliography[keyword={complementaria}, title={Referencias Complementarias}]

\end{document}